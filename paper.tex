\documentclass[runningheads]{llncs}
%
\usepackage{silence}
\usepackage{graphicx}

\begin{document}
%
\title{Nombre Trabajo Final}
%
%\titlerunning{Abbreviated paper title}
% If the paper title is too long for the running head, you can set
% an abbreviated paper title here
%
\author{Juan José Sapuyes Pino\orcidID{0000-1111-2222-3333} \and
    Juan Manuel Pajoy López\orcidID{1111-2222-3333-4444} \and
    Juan Pablo Ortega Medina\orcidID{2222--3333-4444-5555} \and
    Sebastián Rendón Giraldo\orcidID{0000-0001-7822-3173}}
%
\authorrunning{J. Sapuyes et al.}
% First names are abbreviated in the running head.
% If there are more than two authors, 'et al.' is used.
%
\institute{Universidad Nacional de Colombia, Medellín, Colombia.
    \email{\{jsapuyesp,jpajoyl,jportegame,serendongi\}@unal.edu.co}}
%
\maketitle              % typeset the header of the contribution
%
\begin{abstract}
    The abstract should briefly summarize the contents of the paper in
    15--250 words.

    \keywords{Clústering  \and Agrupamiento \and Jerárquico \and Programación Dinámica.}
\end{abstract}

\section{Introducción}
Las empresas dedicadas a los servicios de laboratorio en casa se encuentran
periódicamente con el problema de encontrar la ruta más óptima para sus enfermeros,
que logre minimizar la distancia entre los puntos de recolección de muestras
$P_{r}$. Los enfermeros suelen comenzar el recorrido desde sus casas $P_o$,
a las 4:00 am, hacia los puntos de servicio designados, finalizando en un
punto de entrega de muestras $P_e$ alrededor de las 10:00am.

Para conocer las rutas óptimas para todos los enfermeros, es necesario realizar
un agrupamiento o clústering geográfico de los puntos de recolección de muestras
por zonas, utilizando como centroides a los puntos iniciales $P_o$,
que representan las casas de los enfermeros encargados de tomar la muestra.
Esto beneficia tanto a las empresas, pues representa un menor consumo de
combustible y posiblemente un mayor número de servicios por día, como a los
enfermeros, que tienen que recorrer menores distancias para realizar los servicios
y tienen una ruta más consistente todos los días.

%
% ---- Bibliography ----
%
% BibTeX users should specify bibliography style 'splncs04'.
% References will then be sorted and formatted in the correct style.
%
\bibliographystyle{splncs04}
\bibliography{bibliography}
\end{document}