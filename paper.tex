\documentclass[runningheads]{llncs}
%
\usepackage{silence}
\usepackage{graphicx}
\usepackage{mathtools}
\DeclarePairedDelimiter{\ceil}{\lceil}{\rceil}
\DeclarePairedDelimiter{\floor}{\lfloor}{\rfloor}
\begin{document}
%
\title{Nombre Trabajo Final}
%
%\titlerunning{Abbreviated paper title}
% If the paper title is too long for the running head, you can set
% an abbreviated paper title here
%
\author{Juan José Sapuyes Pino\orcidID{0000-1111-2222-3333} \and
    Juan Manuel Pajoy López\orcidID{1111-2222-3333-4444} \and
    Juan Pablo Ortega Medina\orcidID{2222--3333-4444-5555} \and
    Sebastián Rendón Giraldo\orcidID{0000-0001-7822-3173}}
%
\authorrunning{}
% First names are abbreviated in the running head.
% If there are more than two authors, 'et al.' is used.
%
\institute{Universidad Nacional de Colombia, Medellín, Colombia.
    \email{\{jsapuyesp,jpajoyl,jportegame,serendongi\}@unal.edu.co}}
%
\maketitle              % typeset the header of the contribution
%
\begin{abstract}
    The abstract should briefly summarize the contents of the paper in
    15--250 words.

    \keywords{Clústering  \and Agrupamiento \and Jerárquico \and Programación Dinámica.}
\end{abstract}

\section{Introducción}
Las empresas dedicadas a los servicios de laboratorio en casa se encuentran
periódicamente con el problema de encontrar la ruta más óptima para sus enfermeros,
que logre minimizar la distancia entre los puntos de recolección de muestras
$P_{r}$. Los enfermeros suelen comenzar el recorrido desde sus casas $P_{o}$,
a las 4:00 am, hacia los puntos de servicio designados, finalizando en un
punto de entrega de muestras $P_{e}$ alrededor de las 10:00am.

Para conocer las rutas óptimas para todos los enfermeros, es necesario realizar
un agrupamiento o clústering geográfico de los puntos de recolección de muestras
por zonas \cite{bard11}, según la cantidad de enfermeros domiciliarios $K$,
encargados de tomar las muestras. Esto beneficia tanto a las empresas,
pues representa un menor consumo de combustible y posiblemente un mayor número
de servicios por día, como a los enfermeros, que tienen que recorrer menores
distancias para realizar los servicios y tienen una ruta más consistente todos
los días.

\section{Planteamiento del problema}
Dado un conjunto de puntos de recolección de muestras $P_{R}$ con coordenadas
$(x, y)$ y una cantidad de enfermeros domiciliarios $K$ para un día de la semana,
necesitamos encontrar $K$ clusters de tamaño mínimo $\floor{\frac{R}{K}}$ y tamaño
máximo $\ceil{\frac{R}{K}}$.

El problema es un subproblema del enrutamiento de flotas, en el cuál debemos encontrar
los grupos geográficos que minimicen las distancias entre clientes, y asignar cada uno
de estos grupos a un enfermero. El problema de hallar la ruta óptima dentro de cada
grupo no se aborda en esta propuesta.
\subsection{Técnicas propuestas por otros autores}
Se han propuesto métodos de clústering geográfico para redes de recolección y entrega
de paquetes, que buscan minimizar el número de vehículos de la flota que
la empresa debe usar, utilizando algoritmos jerárquicos aglomerativos con una función
de distancia que logren minimizar los tiempos (costo) entre clientes de un clúster
\cite{bard11}.
\\
Nazari, et al. \cite{nazari19}, proponen un algoritmo \textit{bottom-up} de clústering
jerárquico con puntos superpuestos con complejidad $O(n^{2})$, agrupando parejas en cada nivel,
encontrando intersecciones entre clústers y creando un nuevo clúster como la unión de estos dos.
\section{Solución}
\subsection{Algoritmo}
\section{Complejidad y comparativa}
%
% ---- Bibliography ----
%
% BibTeX users should specify bibliography style 'splncs04'.
% References will then be sorted and formatted in the correct style.
%
\renewcommand{\refname}{Referencias}
\bibliographystyle{splncs04}
\bibliography{bibliography}
\end{document}