\documentclass[runningheads]{llncs}
%
\usepackage{silence}
\usepackage{graphicx}
\usepackage{mathtools}
\DeclarePairedDelimiter{\ceil}{\lceil}{\rceil}
\DeclarePairedDelimiter{\floor}{\lfloor}{\rfloor}
\begin{document}
%
\title{Agrupamiento jerárquico balanceado para el problema de enrutamiento de flotas}
%
%\titlerunning{Abbreviated paper title}
% If the paper title is too long for the running head, you can set
% an abbreviated paper title here
%
\author{Juan José Sapuyes Pino \and
    Juan Manuel Pajoy López \and
    Juan Pablo Ortega Medina \and
    Sebastián Rendón Giraldo}
%
\authorrunning{}
% First names are abbreviated in the running head.
% If there are more than two authors, 'et al.' is used.
%
\institute{Universidad Nacional de Colombia, Medellín, Colombia.
    \email{\{jsapuyesp,jpajoyl,jportegame,serendongi\}@unal.edu.co}}
%
\maketitle              % typeset the header of the contribution
%
\begin{abstract}
    The abstract should briefly summarize the contents of the paper in
    15--250 words.
    \keywords{Clústering  \and Agrupamiento \and Jerárquico.}
\end{abstract}

\section{Introducción}
Existen laboratorios que prestan el servicio de toma de muestras en casa,
un servicio bastante útil para personas que por alguna dificultad no pueden desplazarse
hasta el lugar donde se toman las muestras, estas muestras se toman y son transportadas
al laboratorio a bajas temperaturas, para poder preservarlas. Sin embargo, los laboratorios
que prestan estos servicios se encuentran con un gran problema, ¿cuál es la ruta óptima para
la persona que toma la muestra? Esto es un problema bastante recurrente para los laboratorios
y es de vital importancia resolverlo, pues las muestras se encuentran en neveras portátiles,
las cuales preservan las bajas temperaturas, pero no durante mucho tiempo.  Estos recorridos se
realizan en horas de la mañana, tomando como punto de partida el hogar del domiciliario $P_{0}$,
este debe visitar los diferentes lugares designados para la recolección de muestras $P_{r}$ y
finalizar el recorrido en el laboratorio $P_{e}$.
\\
Para determinar las rutas optimas para los domiciliarios, se realiza un proceso de agrupamiento
o clústering geográfico de los diferentes puntos de recolección $P_{r}$ por zonas \cite{bard11},
para esto se tiene en cuenta la cantidad de domiciliarios disponibles $K$, los cuales se encargarán
de tomar las muestras.
\\
Este procedimiento no sólo beneficiará a los laboratorios que presten estos servicios representando
menores costos logísticos y probablemente un mayor número de servicios diarios, sino también a los
domiciliarios encargados de tomar las muestras, pues tendrán rutas más consistentes diariamente y
recorrerán menores distancias.


\section{Planteamiento del problema}
Dado un conjunto de puntos de recolección de muestras $P_{R}$ con coordenadas
$(x, y)$ y una cantidad de enfermeros domiciliarios $K$ para un día de la semana,
necesitamos encontrar $K$ clusters de tamaño mínimo $\floor{\frac{R}{K}}$ y tamaño
máximo $\ceil{\frac{R}{K}}$.

El problema es un subproblema del enrutamiento de flotas, en el cuál debemos encontrar
los grupos geográficos que minimicen las distancias entre clientes, y asignar cada uno
de estos grupos a un enfermero. El problema de hallar la ruta óptima dentro de cada
grupo no se aborda en esta propuesta.
\subsection{Técnicas propuestas por otros autores}
Se han propuesto métodos de clústering geográfico para redes de recolección y entrega
de paquetes, que buscan minimizar el número de vehículos de la flota que
la empresa debe usar, utilizando algoritmos jerárquicos aglomerativos con una función
de distancia que logren minimizar los tiempos (costo) entre clientes de un clúster
\cite{bard11}.
\\
Nazari, et al. \cite{nazari19}, proponen un algoritmo \textit{bottom-up} de clústering
jerárquico con puntos superpuestos con complejidad $O(n^{2})$, agrupando parejas en cada nivel,
encontrando intersecciones entre clústers y creando un nuevo clúster como la unión de estos dos.
\section{Método}
Para la creación de los grupos, se va a utilizar un algoritmo aglomerativo, que implica que se 
van a calcular los puntos más cercanos de todo el conjunto de puntos, que se irán juntando poco a poco hasta formar grupos más grandes con la cantidad deseada
. Esta distancia se va a calcular teniendo en cuenta el algoritmo de divide y vencerás para la selección
de estos puntos, armando parejas muy juntas. Una vez
se unan los puntos se generará un nuevo punto, que será el centroide de los puntos selecionados para integrarlo
nuevamente a la lista de puntos y volver a calcular los puntos más cercanos y asi ir generando los puntos uniendo
cada punto medio. En el caso en el que no se pueda armar un cluster debido a que cuando se unen varios cluster la
suma de la cantidad de puntos sea mayor a la cantidad máxima de puntos permitida, "aqui va la explicación de lo que está haciendo pablo"
.El problema exige que cada clúster tenga una cantidad determinada de puntos, por lo que es necesario
limitar la cantidad de puntos que se pueden agregar a un clúster

Al finalizar la seleción de los clusters, se hará la asignacion de cada uno de los grupos a cada uno de los domiciliarios, que se hace simplemente 
\subsection{Algoritmo}
\section{Resultados}
\section{Conclusiones}
%
% ---- Bibliography ----
%
% BibTeX users should specify bibliography style 'splncs04'.
% References will then be sorted and formatted in the correct style.
%
\renewcommand{\refname}{Referencias}
\bibliographystyle{splncs04}
\bibliography{bibliography}
\end{document}